\section{Motivation}

Security vulnerabilities may incur severe lost to users.
EXAMPLE!!!

The goal of taint analysis is to find external inputs
to a program that are used without sanitizing checks.
Functions performing sanitizing checks are called
\textit{sanitizer}. Developers declare sanitizers before
analysis execution.
A program input location that may return an unsafe
value is called a \textit{source}.
Taint analysis proceeds by first tagging values from sources
as \textit{tainted} (e.g. return value from a system call).
Once tagged, tainted inputs are tracked through
\textit{taint propagation}: a value that results from an
operation involving a tainted input as operand is also tainted.
A tainted value becomes safe after it has been sanitized.
\textit{Sinks} are program locations that may use tainted
values (e.g. a function call). The analysis emits a warning
when a tainted value is used at a sink.  