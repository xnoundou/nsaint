
\documentclass[11pt]{article} % use larger type; default would be 10pt

\usepackage[utf8]{inputenc} % set input encoding (not needed with XeLaTeX)

%%% PAGE DIMENSIONS
\usepackage{geometry} % to change the page dimensions
\geometry{a4paper} % or letterpaper (US) or a5paper or....
\usepackage{amsmath,amssymb,graphicx,algorithmic,xspace,url,latexsym}
\usepackage{graphicx} % support the \includegraphics command and options
\usepackage{url}
\usepackage[parfill]{parskip} % Activate to begin paragraphs with an empty line rather than an indent

\usepackage{xcolor}
\definecolor{forestgreen}{RGB}{2,160,70}    
\definecolor{mediumblue}{RGB}{7,43,205}    
\definecolor{firebrickred}{RGB}{178,34,34}
\definecolor{listingray}{gray}{0.9}
\definecolor{lbcolor}{rgb}{0.9,0.9,0.9}

\usepackage{hyperref}
\hypersetup{
    colorlinks,
    citecolor=forestgreen,
    filecolor=black,
    linkcolor=red,
    urlcolor=firebrickred,
}
\usepackage{paralist}
%---------------------------------------------------------------

%%% PACKAGES
\usepackage{booktabs} % for much better looking tables
\usepackage{array} % for better arrays (eg matrices) in maths
\usepackage{paralist} % very flexible & customisable lists (eg. enumerate/itemize, etc.)
\usepackage{verbatim} % adds environment for commenting out blocks of text & for better verbatim
\usepackage{subfig} % make it possible to include more than one captioned figure/table in a single float
\usepackage{epsfig}

% These packages are all incorporated in the memoir class to one degree or another...
%---------------------------------------------------------------

%%% HEADERS & FOOTERS
\usepackage{fancyhdr} % This should be set AFTER setting up the page geometry
\pagestyle{fancy} % options: empty , plain , fancy
\renewcommand{\headrulewidth}{0pt} % customise the layout...
\lhead{}\chead{}\rhead{}
\lfoot{}\cfoot{\thepage}\rfoot{}
%---------------------------------------------------------------

%%% SECTION TITLE APPEARANCE
\usepackage{sectsty}
\allsectionsfont{\sffamily\mdseries\upshape} % (See the fntguide.pdf for font help)
%\renewcommand \thesection{\Alph{section}}
% (This matches ConTeXt defaults)
%---------------------------------------------------------------

%%% ToC (table of contents) APPEARANCE
\usepackage[nottoc,notlof,notlot]{tocbibind} % Put the bibliography in the ToC
\usepackage[titles,subfigure]{tocloft} % Alter the style of the Table of Contents
\renewcommand{\cftsecfont}{\rmfamily\mdseries\upshape}
\renewcommand{\cftsecpagefont}{\rmfamily\mdseries\upshape} % No bold!

%---------------------------------------------------------------

% enable customized headers and footers
\pagestyle{myheadings}
\markright{\textsc{saint} user's manual}
%%% define footer
%\cofoot{JUnit 4 Tutorial}

%---------------------------------------------------------------


\newcommand{\Cobertura}{\textbf{Cobertura}}
\newcommand{\JUnit}{\textbf{JUnit 4}}
\newcommand{\UUT}{Unit Under Test}
\newcommand{\Site}[1]{ \textbf{\textit{#1}} }

\usepackage{listings}
\usepackage{color}
%---------------------------------------------------------------
\newcommand{\saint}{\texttt{\textsc{saint}}\xspace}
\newcommand{\software}[2]{\texttt{\textbf{#1}, version #2}\xspace}
\newcommand{\softwarenov}[1]{\texttt{\textbf{#1}}\xspace}
\newcommand{\tool}[1]{\texttt{#1}\xspace}
\newcommand{\command}[1]{\textcolor{mediumblue}{\texttt{\textbf{"#1"}}}\xspace}
\newcommand{\script}[1]{\texttt{#1}\xspace}

\newcommand{\env}[1]{\textcolor{firebrickred}{\text{#1}}\xspace}
\newcommand{\envout}[1]{\textcolor{firebrickred}{\$\text{#1}}\xspace}

\newcommand{\mycheckmark}[1]{\textcolor{#1}{$\checkmark$}}

\newcommand{\company}[1]{\textbf{#1}\xspace}
\newcommand{\diplinf}{\textsc{Dipl.-Inf.}\xspace}

%---------------------------------------------------------------

%%% END Article customizations
\begin{document}

\title{ \textcolor{mediumblue}{\textsc{Saint: Simple Static Taint Analysis Tool}\\
				\textcolor{mediumblue}{User's manual}}}
\author{ \textcolor{mediumblue}{\textsc{Dipl.-Inf. Xavier NOUMBISSI NOUNDOU}}\\
\textit{\textcolor{mediumblue}{\textsc{xavier.noumbis@gmail.com}}}}

\maketitle

\bigskip 
%---------------------------------------------------------------

\tableofcontents

\newpage

\section{Author's Biography}
\begin{figure}[htpb]
\includegraphics[scale=0.5]{XavierNOUMBISSI-NOUNDOU-2}
\caption{The author Xavier NOUMBISSI NOUNDOU}~\label{fig:xaviernoumbis}
\end{figure}

\company{Xavier NOUMBISSI NOUNDOU} is from CAMEROON and holds the title Diplom-Informatiker
[\textsc{Dipl.-Inf.}]~\footnote{\url{http://www.inb.uni-luebeck.de/~boehme/diplinf.html}}
(roughly equivalent to a canadian Master's degree in Computer Science)
of the \company{University of Bremen}~\footnote{\url{http://www.uni-bremen.de}}
in Germany.

After his Diplom-Informatiker degree, he worked 18 months
as Software Developer for 
\company{\textsc{Siemens} Healthcare}~\footnote{\url{http://www.healthcare.siemens.com}}
in Erlangen (Germany).

After \company{\textsc{Siemens}}, Xavier started his doctoral
research in Program Analysis and Software Engineering in the
\company{Watform Lab} at the 
\company{University of Waterloo}~\footnote{\url{http://www.uwaterloo.ca}}
(Waterloo, Ontario, Canada).

From January~2012 to August~2012, Xavier worked in the Java J9-JIT
compilation team of \company{IBM Toronto Lab.} in Markham (Ontario, Canadat)
as a graduate intern in compiler optimization.

Xavier is professionally proficient in the French, English and
German languages.

For his \diplinf degree, Xavier worked on the automatic
generation of test cases for reactive systems. The algorithms
he developed are used by the German company
\company{Verified Systems International GmbH}~\footnote{\url{http://www.verified.de}}.
The title of his diplom-informatiker thesis was
''\texttt{Statistical test cases generation for reactive systems}''.

Xavier currently works on his PhD degree. His research focuses
on program analysis and software engineering. He is the creator
of \saint~\footnote{\url{https://www.github.com/xaviernoumbis/saint}}, which
is a tool to perform static taint analysis on programs written in the
C programming language.

\section{Introduction}
Businesses increasingly use software. This is even more
relevant for companies relying on e-commerce. However,
software is error-prone and contain several bugs. Security
bugs are one of the major problems faced by companies today.
In the worst case, security bugs enable unauthorized users
to gain full control of an application.

My PhD thesis introduces the concept of 
\textcolor{firebrickred}{\textit{tainted paths}} and
describes techniques and algorithms to compute them in
any imperative programming language that uses
pointers (C, C++, Java, etc.). I implemented these
algorithms in \saint.

\saint does not require the developer to annotate
the program under analysis. \saint implements a
flow-sensitive, interprocedural and context-sensitive
analysis that computes tainted paths in C programs
at compile-time.

\section{Installation Instructions}
This section explains how to install \saint on a ''Linux'' machine.
We haven't tested \saint on a ''Windows'' or ''Mac OS'' machine,
but the installation should follow similar steps. 

\subsection{Required Software}
This section enumerates all software that you need to run \saint.
\begin{itemize}
	\item \saint: \url{https://github.com/xaviernoumbis/saint.git}
	
	\item The compiler infrastructure \software{LLVM}{3.3}: \url{http://llvm.org}
	
	\item The DSA pointer analysis \softwarenov{poolalloc}:
	\url{https://github.com/llvm-mirror/poolalloc.git}
\end{itemize}

\subsection{Environment Variables}
Table~\ref{table_env} that shows all environment variables
that you have to define and export in order to successfully
run \saint.

\begin{table*}[!htbp]
\begin{center}
\begin{tabular}{l|l}
{\bf Environment variables}	&	{\bf Description}	\\ \hline \hline
\env{SAINT\_HOME}	&	\saint home folder (e.g. /home/user/saint) 			\\ 
\env{SAINT\_BIN}	&	\saint binaries folder (e.g. \envout{SAINT\_HOME}/bin) 	\\ 
\env{LLVM\_HOME}	&	\tool{llvm} home folder (e.g. /home/user/llvm)			\\ 
\env{LLVM\_LIB}		&	\tool{llvm} compiled libraries folder 					\\
					&    (e.g. \envout{LLVM\_HOME}/build/Release+Asserts/lib)	\\ 
\env{LLVM\_BIN}		&	\tool{llvm} compiled binaries							\\
					&    (e.g. \envout{LLVM\_HOME}/build/Release+Asserts/bin)	\\ 
\env{POOLALLOC}		&	\tool{poolalloc} home folder (e.g. /home/user/poolalloc)\\ 
\env{CLANGLLVM\_BIN}&	\tool{clang+llvm} binaries' folder	\\
					& 	(e.g. /home/user/clang+llvm/bin) \\ \hline	
\end{tabular}
\end{center}
\caption{Table with all environment variables required to install and use \saint}\label{table_env}
\end{table*}

\mycheckmark{mediumblue} You define and export an environment variable \env{ENV\_VAR} by
writing the following commands in your ''\texttt{\$HOME/.bashrc}'' file:
\begin{verbatim}
	ENV_VAR=path_to_folder
	export ENV_VAR
\end{verbatim}

\section{How to Configure ''\tool{clang+llvm}'' for use with \textsc{saint}}
\begin{enumerate}[a)]
       \item Download and unpack \software{clang+llvm}{3.3}.
       
       \item Add the \texttt{bin} folder to your environment variable \env{PATH}.\\
       \mycheckmark{mediumblue} For instance by adding the following line
       in your file ''\texttt{\$HOME/.bashrc}''
       \begin{verbatim}
               PATH=$PATH:$CLANGLLVM_BIN
               export PATH
       \end{verbatim}
-\end{enumerate}


\section{How to Configure ''\tool{LLVM}'' for use with \textsc{saint}}
\begin{enumerate}[a)]	
	\item Open the file ''\texttt{\$LLVM\_HOME/lib/Analysis/Makefile}'' and
	append the string "saint" to the ''DIRS'' variable.
	Following is an excerpt of the file.\\
	
	\begin{verbatim}
##===- lib/Analysis/Makefile -------------------------------*- Makefile -*-===##
#
#                     The LLVM Compiler Infrastructure
#
# This file is distributed under the University of Illinois Open Source
# License. See LICENSE.TXT for details.
#
##===----------------------------------------------------------------------===##

LEVEL = ../..
LIBRARYNAME = LLVMAnalysis
DIRS = IPA saint
BUILD_ARCHIVE = 1

include $(LEVEL)/Makefile.common
	\end{verbatim}		
	
	\item Run the script \script{saint-configure.sh}.
\end{enumerate}

\section{Folder Structure}
The following folders constitute \saint's directory structure:
\begin{enumerate}[1.]
	\item \texttt{bin}: folder with the scripts:
	\begin{itemize}
		\item \script{saint-gen-ir.sh}: generates the LLVM IR for code analyzed by \saint.
		\item \script{saint-run-llvm-opt.sh}: runs LLVM (\texttt{opt} binary) with \saint as plugin.
		\item \script{saint-configure.sh}: configures and compiles \tool{poolalloc}
		      and \tool{LLVM} for \saint.
	\end{itemize}
	
	\item \texttt{benchmarks}: folder with sample
	scripts to run \saint.

	\item \texttt{cfg}: folder with source, sink, and sanitizer configuration files
		
	\item \texttt{projects}: folder with sample projects.	

	\item \texttt{doc}: folder with the manual.
	
	\item \texttt{src}: folder with all \texttt{C++} source files,
	and \tool{Bash} scripts to compile and run \saint.			
\end{enumerate}

\section{How to Compile and Run \textsc{saint}}

\mycheckmark{mediumblue} You need to execute the command
\command{make -f Makefile.saint} within the
folder ''\texttt{\$LLVM\_HOME/lib/Analysis/saint}'' to compile
\saint.

Also, \saint gets compiled when you run it using
the Bash script \script{saint-run-llvm-opt.sh}.

\subsection{Configuration Files}

Configuration files are found in the folder ''\texttt{\$SAINT\_HOME/cfg}''.
There are three configuration files:
\begin{itemize}
	\item \texttt{sources.cfg}: taint sources configuration file\\
		  Each line of the file specifies a function name and an integer ''\texttt{x}''.
		  ''\texttt{x}'' is the argument of the function that is tainted.\\
		  If ''\texttt{x}'' is zero (0), then it is the return value of the function that is tainted.
		  \begin{verbatim}
		  fopen,0
		  fgets,1		  
		  \end{verbatim}
		  The previous lines specify that \texttt{fopen} returns a tainted value, and
		  that \texttt{fgets} taints its first argument. 
	\item \texttt{sinks.cfg}: taint sinks configuration file\\
		  Each line of the file specifies a function name and an integer ''\texttt{y}''.
		  ''\texttt{y}'' is the argument of the function that must not received a tainted value.\\
		  ''\texttt{y}'' is never equal to zero (0).
		  \begin{verbatim}
		  sprintf,1	  
		  \end{verbatim}
		  The previous line denotes that \texttt{sprintf} must not received
		  a tainted value as its first argument (i.e. \texttt{sprintf} is 
		  sensible function.
	\item \texttt{sanitizers.cfg}: sanitizers configuration file\\
		  \saint doesn't yet implement this functionality.
\end{itemize}

\subsection{How to Run \textsc{saint}}
Among others, \saint source folder contains the following two important
\texttt{Bash} scripts:
\begin{itemize}
	\item \texttt{saint-gen-ir.sh}: this script is used to generate and
		merge \tool{llvm} intermediate representation (IR) files. 
	
	\item \texttt{saint-run-opt.sh}: this script is used to run the analysis of
		\saint on the program under analysis.
\end{itemize}

We encourage users to look at the sample scripts in the
folders ''\texttt{projects}'' and ''\texttt{benchmarks}''
to learn how to use \texttt{saint-gen-ir.sh} and
\texttt{saint-run-opt.sh}.
		
\subsection{How to Get Debug Messages from \textsc{saint}}		


\bibliographystyle{plain}
\bibliography{saint-manual}


\end{document}
