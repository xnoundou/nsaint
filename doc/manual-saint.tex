\documentclass[12pt,onecolumn,a4paper]{article}
\usepackage{amsmath,amssymb,graphicx,algorithmic,xspace,url,latexsym}

\usepackage[parfill]{parskip}
\usepackage{paralist}
\usepackage{times}
\usepackage{textcomp}
\usepackage{verbatim}
\usepackage{hyperref}
\usepackage{xcolor}
\usepackage{adjustbox} %To color the backround of the verbatim environment
\usepackage{verbatim} 
\definecolor{shadecolor}{rgb}{.9, .9, .9} %To color the backround of the verbatim environment

%To color the backround of the verbatim environment
\newenvironment{verbatim_color}%
   {\par\noindent\adjustbox{margin=1ex,bgcolor=shadecolor,margin=0ex \medskipamount}\bgroup\minipage\linewidth\verbatim}%
   {\endverbatim\endminipage\egroup}

\hypersetup{
    colorlinks,
    citecolor=black,
    filecolor=black,
    linkcolor=purple,
    urlcolor=black
}

\setlength{\evensidemargin}{-0.1in} \setlength{\oddsidemargin}{-0.1in}
\setlength{\textwidth}{6.5in} \setlength{\textheight}{9.0in}
\setlength{\topmargin}{-0.5in}
%\setlength{\headheight}{0in}

\definecolor{forestgreen}{RGB}{34,139,34}    
\definecolor{mediumblue}{RGB}{0,0,205}    
\definecolor{firebrickred}{RGB}{178,34,34}

\newcommand{\saint}{\texttt{\textsc{saint}}\xspace}
\newcommand{\software}[2]{\texttt{\textbf{#1}, version #2}\xspace}
\newcommand{\softwarenov}[1]{\texttt{\textbf{#1}}\xspace}
\newcommand{\tool}[1]{\texttt{#1}\xspace}
\newcommand{\command}[1]{\textcolor{mediumblue}{\texttt{\textbf{"#1"}}}\xspace}
\newcommand{\script}[1]{\texttt{#1}\xspace}

\newcommand{\env}[1]{\textcolor{firebrickred}{\text{#1}}\xspace}
\newcommand{\envout}[1]{\textcolor{firebrickred}{\$\text{#1}}\xspace}

\newcommand{\mycheckmark}[1]{\textcolor{#1}{$\checkmark$}}

%---------------------------------------------------------------
%%% HEADERS & FOOTERS
\usepackage{fancyhdr} % This should be set AFTER setting up the page geometry
\pagestyle{fancy} % options: empty , plain , fancy
\renewcommand{\headrulewidth}{0pt} % customise the layout...
\lhead{}\chead{}\rhead{}
\lfoot{}\cfoot{\thepage}\rfoot{}

%---------------------------------------------------------------
%%% SECTION TITLE APPEARANCE
\usepackage{sectsty}
\allsectionsfont{\sffamily\mdseries\upshape} % (See the fntguide.pdf for font help)
%\renewcommand \thesection{\Alph{section}}
% (This matches ConTeXt defaults)
%---------------------------------------------------------------

\pagestyle{myheadings}
\markright{Junit 4 Tutorial}

%Remove widows and orphants
\clubpenalty = 10000
\widowpenalty = 10000
\displaywidowpenalty = 10000 

\sloppypar
\begin{document}
\pagestyle{empty}

\title{\textsc{Saint: Simple Static Taint Analysis Tool\\
				User's manual}}
\author{Xavier NOUMBISSI NOUNDOU}

\begin{center}
\begin{LARGE}
\textbf{ \textsc{Saint: Simple Static Taint Analysis Tool\\
				User's manual}}\\
\end{LARGE}
\begin{large}
\vspace{0.3cm}
by\\
\vspace{0.3cm}
\texttt{ \bf \textit{Dipl.-Inf. Xavier Noumbissi Noundou\\
xavier.noumbis@gmail.com}\\
\url{https://sites.google.com/site/xaviernoumbis}}\\
\vspace{0.3cm}
\today{}\\
\vspace{1.3cm}
\end{large}
\end{center}

\tableofcontents

\section{Abstract}
Businesses increasingly use software. This is even more
relevant for companies relying on e-commerce. However,
software is error-prone and contain several bugs. Security
bugs are one of the major problems faced by companies today.
In the worst case, security bugs enable unauthorized users
to gain full control of an application.

My PhD thesis introduces the concept of 
\textcolor{firebrickred}{\textit{tainted paths}} and
describes techniques and algorithms to compute them in
any imperative programming language that uses
pointers (C, C++, Java, etc.). I implemented these
algorithms in \saint.

\saint does not require the developer to annotate
the program under analysis. \saint implements a
flow-sensitive, interprocedural and context-sensitive
analysis that computes tainted paths in C programs
at compile-time.

\section{Installation Instructions}
This section of the manual explains how to install \saint
on a ''Linux'' machine. We have not tested \saint on a ''Windows''
or ''Mac OS'' machine, but the installation should follow similar steps. 

\subsection{Required Software}
This section enumerates all software that you need to run \saint.
\begin{enumerate}[a)]
	\item \saint: \url{https://github.com/xaviernoumbis/saint.git}
	
	\item The compiler infrastructure \software{LLVM}{3.3} (\url{http://llvm.org})
	
	\item The DSA pointer analysis \softwarenov{poolalloc}
	(\url{https://github.com/llvm-mirror/poolalloc.git}).
\end{enumerate}

\subsection{Environment Variables}
Table~\ref{table_env} that shows all environment variables
that you have to define and export in order to successfully
run \saint.

\begin{table*}[!htbp]
\begin{center}
\begin{tabular}{|r|l|} \hline
{\bf Environment variables}	&	{\bf Description}	\\ \hline \hline
\env{SAINT\_HOME}	&	\saint home folder (e.g.: /home/user/saint) 	\\ \hline
\env{SAINT\_BIN}	&	\saint binaries folder (e.g.: /home/user/saint/bin) 	\\ \hline
\env{LLVM\_HOME}	&	\tool{llvm} home folder (e.g.: /home/user/llvm)		\\ \hline
\env{LLVM\_LIB}		&	\tool{llvm} compiled libraries folder 					\\
					&    (e.g.: \envout{LLVM\_HOME}/build/Release+Asserts/lib)	\\ \hline
\env{LLVM\_BIN}		&	\tool{llvm} compiled binaries					\\
					&    (e.g.: \envout{LLVM\_HOME}/build/Release+Asserts/bin)	\\ \hline	
\env{POOLALLOC}		&	\tool{poolalloc} home folder (e.g.: /home/user/poolalloc) \\ \hline		
\env{CLANGLLVM\_BIN}&	\tool{clang+llvm} binaries' folder	\\
					& 	(e.g.: /home/user/clang+llvm/bin) \\ \hline	
\end{tabular}
\end{center}
\caption{Table with all environment variables required to install and use \saint}\label{table_env}
\end{table*}

\mycheckmark{mediumblue} You define and export an environment variable \env{ENV\_VAR} by
writing the following commands in your ''\texttt{.bashrc}'' file:
\begin{verbatim}
	ENV_VAR=path_to_folder
	export ENV_VAR
\end{verbatim}

\subsection{How to Configure ''\tool{LLVM}'' for use with \textsc{saint}}
\begin{enumerate}[a)]	
	\item Open the file ''\texttt{\$LLVM\_HOME/lib/Analysis/Makefile}'' and
	append the string "saint" to the ''DIRS'' variable.
	Following is an excerpt of the file.\\
	
	\begin{verbatim}
##===- lib/Analysis/Makefile -------------------------------*- Makefile -*-===##
#
#                     The LLVM Compiler Infrastructure
#
# This file is distributed under the University of Illinois Open Source
# License. See LICENSE.TXT for details.
#
##===----------------------------------------------------------------------===##

LEVEL = ../..
LIBRARYNAME = LLVMAnalysis
DIRS = IPA saint
BUILD_ARCHIVE = 1

include $(LEVEL)/Makefile.common
	\end{verbatim}		
	
	\item Run the script \script{saint-configure.sh}.
\end{enumerate}

\section{Folder Structure}
The following folders constitute \saint's directory structure:
\begin{enumerate}[1)]
	\item \texttt{bin}: folder with the scripts:
	\begin{enumerate}[a)]
		\item \script{saint-gen-ir.sh}: generates the LLVM IR for code analyzed by \saint.
		\item \script{saint-run-llvm-opt.sh}: runs LLVM (\texttt{opt} binary) with \saint as plugin.
		\item \script{saint-configure.sh}: configures and compiles \tool{poolalloc}
		      and \tool{LLVM} for \saint.
	\end{enumerate}
	
	\item \texttt{benchmarks}: folder with sample
	scripts to run \saint.

	\item \texttt{cfg}: folder with source, sink, and sanitizer configuration files
		
	\item \texttt{projects}: folder with sample projects.	

	\item \texttt{doc}: folder with the manual.
	
	\item \texttt{src}: folder with all \texttt{C++} source files,
	and \tool{Bash} scripts to compile and run \saint.			
\end{enumerate}

\section{Compiling and Running \textsc{saint}}

\mycheckmark{mediumblue} You need to execute the command
\command{make -f Makefile.saint} within the
folder ''\texttt{\$LLVM\_HOME/lib/Analysis/saint}'' to compile
\saint.

Also, \saint gets compiled when you run it using
the Bash script \script{saint-run-llvm-opt.sh}.

\subsection{Configuration Files}

Configuration files are found in the folder ''\texttt{\$SAINT\_HOME/cfg}''.
There are three configuration files:
\begin{enumerate}[1)]
	\item \texttt{sources.cfg}: taint sources configuration file\\
		  Each line of the file specifies a function name and an integer ''\texttt{x}''.
		  ''\texttt{x}'' is the argument of the function that is tainted.\\
		  If ''\texttt{x}'' is zero (0), then it is the return value of the function that is tainted.
		  \begin{verbatim}
		  fopen,0
		  fgets,1		  
		  \end{verbatim}
		  The previous lines specify that \texttt{fopen} returns a tainted value, and
		  that \texttt{fgets} taints its first argument. 
	\item \texttt{sinks.cfg}: taint sinks configuration file\\
		  Each line of the file specifies a function name and an integer ''\texttt{y}''.
		  ''\texttt{y}'' is the argument of the function that must not received a tainted value.\\
		  ''\texttt{y}'' is never equal to zero (0).
		  \begin{verbatim}
		  sprintf,1	  
		  \end{verbatim}
		  The previous line denotes that \texttt{sprintf} must not received
		  a tainted value as its first argument (i.e. \texttt{sprintf} is 
		  sensible function.
	\item \texttt{sanitizers.cfg}: sanitizers configuration file\\
		  \saint doesn't yet implement this functionality.
\end{enumerate}

\subsection{Running \textsc{saint}}
Among others, \saint source folder contains the following two important
\texttt{Bash} scripts:
\begin{enumerate}[1)]
	\item \texttt{saint-gen-ir.sh}: this script is used to generate and
		merge \tool{llvm} intermediate representation (IR) files. 
	
	\item \texttt{saint-run-opt.sh}: this script is used to run the analysis of
		\saint on the program under analysis.
\end{enumerate}

We encourage users to look at the sample scripts in the
folders ''\texttt{projects}'' and ''\texttt{benchmarks}''
to learn how to use \texttt{saint-gen-ir.sh} and
\texttt{saint-run-opt.sh}.
		
\subsection{Getting debug messages from \textsc{saint}}		
		
%%%%%%%%%%%%%%%%%%%%%%%%%%%%%%%%%%%%%%%%%%%%%%%%%%%%%%%%%%%%%%%%%%%%%%%%%%%%%%%%%%

\bibliographystyle{plain}
\bibliography{manual-saint}

\end{document}
