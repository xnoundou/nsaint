\section{Conclusion}\label{sec:conclusion}

In this project report, we have presented an interprocedural
taint analysis that comes in two flavors: a context-sensitive
and a context-insensitive version.

Due to time constraints, we could only implement the
LLVM infrastructure for hosting our analysis algorithms.
We are going to finish the implementation in the following
weeks as we intend to submit our work for publication
in a conference.

The presented analysis does not handle the use of
\textit{sanitizer functions} in the analyzed code.
In future work, we plan to generate kill sets at
program points whenever an adequate sanitizer will
be encountered.

Our current analysis does not track tainted fields of
\texttt{struct} and arrays. We plan to add this feature
in future work.

Given the staged nature of the presented analysis, and its
usage of a summary table that stores newer analysis results
(i.e. new information from successive analyzes run are stored
in the summary table), we believe that an integration in IDE
would be possible and plan this as future work.